\documentclass[8pt, twocolumn]{article}
\input{header.tex}

\usepackage{fancyhdr}
\pagestyle{fancy}

\lhead{Physics Reference Sheet}
\rhead{Joss Steward}

\renewcommand{\headrulewidth}{0.4pt}
\renewcommand{\footrulewidth}{0.4pt}

\begin{document}

\section{General Equations}
	Kinematic Equations:
	\[
		\begin{aligned}
		X(t) &= X_i + V_{xi} \times t + \frac{a_x \times t ^ 2}{2} \\
		V(t) &= V_{xi} + a_x \times t \\
		V_x^2 &= V_{xi}^2 + 2a_x \times (\Delta X)
		\end{aligned}		
	\]

\subsection{Springs (X = displacement from rest position)}
	\[		
		F_x = -kx
	\]
\subsection{Friction}
	\[
		\begin{aligned}
		f_{smax} &= \mu_s \cdot F_N \\
		f_k &= \mu_k \cdot F_N \\
		F_N &= mg \\
		f_{smax} &\geq f_k 
		\end{aligned}		
	\]

\subsection{Dot Products}
	\[
		\vec{A} \cdot \vec{B} = AB \times \cos{\Theta}
	\]
	\[
		\vec{A} \cdot \vec{B} = \begin{cases}
						0 & \text{When } \vec{A} \text{ and } \vec{B} \text{ are perpendicular} \\
						AB & \text{When } \vec{A} \text{ and } \vec{B} \text{ are parrallel}
					 \end{cases}
	\]
	\[
	\begin{aligned}
		\vec{A} \cdot \vec{A} &= A^2 \\
		\vec{A} \cdot \vec{B} &= \vec{B} \cdot \vec{A} \\
		(\vec{A} + \vec{B}) \cdot \vec{C} &= (\vec{A} \cdot \vec{C}) + (\vec{B} \cdot \vec{C})
	\end{aligned}
	\]

\subsection{Conservative and Nonconservative forces}
	Conservative Forces - Path taken is irrelevant \newline
	Nonconservative Forces - Path matters
	\[\oint_c (\vec{F}) \cdot d\vec{l} = W_{total}\]
	C is the closed path length.  This problem can be broken into pieces and solved.

\subsection{Linear Motion}
	Displacement:
	\[\Delta X\]

	Velocity:
	\[V_x = \frac{dx}{dt}\]

	Acceleration:
	\[a_x = \frac{dv_x}{dt} = \frac{d^2x}{dt^2}\]

	Constant-acceleration equations:
	\[\begin{aligned}
		&V_x = V_{xi} + a_xt \\
		&\Delta X = V_{x avg}\Delta t \\
		&V_{x avg} = \frac{1}{2}(V_{xi} + V_x \\
		&X = X_i + V_{xi}t + \frac{1}{2}(a_xt^2) \\
		&V_x^2 = V_{xi}^2 + 2a_x\Delta X
	\end{aligned}\]

	Force: $F_x$

	Mass: $m$

	Work: $dW = F_xdx$

	Kinetic Energy: $K = \frac{1}{2}mv^2$

	Power: $P = F_x V_x$

	Momentum: $p_x = mV_x$

	Newton's Second Law:
	\[F_{x net} = ma_x = \frac{dp_x}{dt}\]

\subsection{Rotational Motion}
	Angular Displacement:
	\[\Delta \Theta\]

	Angular Velocity:
	\[\omega = \frac{d\Theta}{dt}\]

	Angular Acceleration:
	\[\alpha = \frac{d\omega}{dt} = \frac{d^2\Theta}{dt^2}\]

	Constant Angular Acceleration equations:
	\[\begin{aligned}
		&\omega = \omega_i + \alpha t \\
		&\Delta \Theta = \omega_{avg}\Delta t \\
		&\omega_{avg} = \frac{1}{2}(\omega_i + \omega) \\
		&\Theta = \Theta_i + \omega_i t + \frac{1}{2}\alpha t^2 \\
		&\omega^2 = \omega_i^2 + 2\alpha \Delta \Theta
	\end{aligned}\]

	Torque: $\tau$

	Moment of inertia: $I$

	Work: $dW = \tau d \Theta$

	Kinetic Energy: $K = \frac{1}{2} I \omega^2$

	Power: $P = \tau \omega$

	Angular Momentum: $L = I\omega$

	Newton's Second Law:
	\[\tau_{net} = I\alpha = \frac{dL}{dt}\]

\section{General Maths}
	Quadratic Equation: 
	\[
		X = \frac{-b \pm \sqrt{b^2 - 4ac}}{2a}
	\]
	Vector Composition:
	\[
		\vec{A} = A_x \uv{i} + A_y \uv{j} + A_z \uv{k}
	\]
	2D Vector Decomposition:
	\[
		\begin{aligned}
			V_y &= \bar{V} \times \sin{\Theta} \\
			V_x &= \bar{V} \times \cos{\Theta} \\
			\bar{V} &= \sqrt{V_x^2 + V_y^2}
		\end{aligned}
	\]
	Standard Units:
	\begin{itemize}
		\item Time $\Rightarrow $ Seconds (s)
		\item Mass $\Rightarrow $ Kilograms (kg)
		\item Length $\Rightarrow $ Meter (m)
		\item Force $\Rightarrow $ Newton (N) $ = \frac{1 kg / 1 m}{1 s^2} $
		\item $ g = 9.81 m/s^2 $
	\end{itemize}
	Geometry:
	\[
		\Theta = \tan^{-1}{\frac{opp}{hyp}}
	\]

\section{Calculus}
	General Rules:
	\begin{enumerate}
		\item As long as $ n \neq -1 $
			\[
				\int t^n dt = \frac{t^{n+1}}{n+1} + C
			\]
		\item For when $ n = 1 $
			\[
				\int t^{-1} dt = \ln{t} + C
			\]
	\end{enumerate}

\section{Work-Kinetic Energy Theorum}
	W-KE Equations:
	\[\begin{aligned}
		W_{total} &= \Delta K \\
		K &= \frac{1}{2}mv^2 \\
		F_{netx} &= \frac{1}{2}mv_f^2 - \frac{1}{2}mv_i^2 \\
		W_{total} &= \Delta K 
	\end{aligned}\]
	\newline
	Kinematics:
	\[\begin{aligned}
		V_f^2 &= V_i^2 + 2a_x\Delta X \\
		a_x &= (V_f^2 - V_i^2) \times \frac{1}{2 \Delta X}
	\end{aligned}\]
	\newline
	Solving Problems:
	\begin{enumerate}
		\item Draw everything at initial and final positions
		\item Add axis (x,y,z)
		\item Draw arrows, initial and final velocities, $\Delta K$, etc
		\item Add force arrows to inital position drawing
		\item Calculate $W_{total} = \Delta K$, $F_{net} d = etc$
	\end{enumerate}

\section{Examples}
\subsection{Kinematics and Integration}

	The $a_x$ of a rocket is given by $ a_x = bt $ where $b$ is a positive constant. Find the position function.
	\newline
	Givens:
	\[ a_x = bt \]
	\[ X = X_0 + V_x t = V_{xi} \text{ at } t = 0 \]
	\newline
	Find the solution:
	\[V_x(t) - V_x(0) = \int_{0}^{t}(bt) \mathrm{d}t = (\frac{1}{2} bt^2)\mid_{0}^{t}\]
	\[V_x(t) = V_{xi} + \frac{1}{2} bt^2\]
	\[X(t) - X(0) = \int_{0}^{t}(V_{xi} + \frac{1}{2}bt^2) \mathrm{d}t\]
	\[X(t) = X_i + V_{xi} t + (\frac{1}{2}b + \frac{1}{3}t^3)\min_{0}^{t}\]
	Solution:
	\[X(t) = X_i + V_{xi}t + \frac{1}{6}bt^3\]

\subsection{Projectile Motion}
	Governing Equations:
	\[V_{xi} = V_i \cos{\Theta}\]
	\[V_{yi} = V_i \sin{\Theta}\]
	\[V_x = V_{xi} \text{ (Assuming no drag) }\]
	\[V_y = V_{yi} - gt\]
	\newline
	Derived Equations:
	\[X(t) = X_i + V_{xi} t\]
	\[Y(t) = Y_i + V_{yi} t + \frac{1}{2} (-g) t^2\]
	\[Y(X) = \frac{V_{yi}}{V_{xi}}X - \frac{g}{2 \times V_{xi}^2} X^2\]

\subsection{Circular Motion}
	\[a_c = \frac{v^2}{r} \text{ Inward } a \text{ needed for const. circ. motion.}\]
	\[v = \sqrt{a_c r}\]
	\[v = \frac{2 \pi r}{t} \text{ t = time for 1 rev.}\]
	\[F = ma = m \frac{v^2}{r} \text{ if } a = a_c\]

\subsection{Summing Vectors}
	Givens:
	\[\vec{F_a} = 40.0 \text{N at } 45\deg\]
	\[\vec{F_b} = 30.0 \text{N at } 37\deg\]
	\newline
	Find sum of vectors:
	\[\vec{F_t} = \sum(F_x) \uv{i} + \sum(F_y) \uv{j}\]
	\[\sum F_{tx} = 40 \cos (45 \deg) + 30 \cos (37 \deg) = 52.3 \text{N}\]
	\[\sum F_{tx} = 40 \sin (45 \deg) + (- 30 \sin (37 \deg)) = 10.2 \text{N}\]
	\[\begin{aligned}
	\vec{F_t} &= \sqrt{10.2^2 + 52.3^2} \text{ in } \Theta \\
	&= \tan^{-1}\frac{10.2}{52.3} = 53.3 \text{N at }11\deg \text{ towards A}\end{aligned}\]

\subsection{Angle when g overcomes friction}
	A coin is resting on an incline of increasing angle.  At what angle does it begin to slide?
	\newline
	Begins to slide when $F_x = F_{fric}$
	\[\begin{aligned}
	&\sum F_y = ma_y = 0 = F_n - mg\cos\Theta \\
	&\sum F_x = mg\sin\Theta - f_s = 0 \end{aligned}\]
	\[\begin{aligned}
	mg\sin\Theta - \mu_s F_n &= 0 \\
	mg\sin\Theta - \mu_s (mg\cos\Theta) &= 0 \\
	mg\sin\Theta &= \mu_s mg\cos\Theta \end{aligned}\]
	Coin starts to slide:
	\[\frac{\sin\Theta}{\cos\Theta} = \tan(\Theta_max) = \mu_s\]

\subsection{Speed at bottom of incline}
	A block is sliding down an incline.  It slides 1.5m, what is it's final speed?
	\newline
	Givens:
	\[\Theta = 60 \deg\]
	\[V_i = 0 \text{m/s}\]
	\[\mu_k = \mu_s = 0\]
	\newline
	Set-up Equations:
	\[\begin{aligned}
	&W = \Delta K \\
	&K = \frac{1}{2}mv^2 \\
	&76 \text{Joules} = \frac{1}{2}mv_f^2 - 0\end{aligned}\]
	\newline
	Work Done:
	\[\begin{aligned}
	&W = F_x \Delta X \\
	&F_x = \sum F_x = F_g = mg\sin(60 \deg) \\
	&W = (1.5\text{m})(6.0\text{kg})(9.81 m/s^2) \sin(60 \deg) = 76 \text{Joules} \\
	&V_{final} = \sqrt{\frac{2 \times 76}{m}} = 5.0m/s
	\end{aligned}\]

\subsection{Potential Energy Example}
	$\Delta U$ = Work done by a conservative force. \newline
	\[\begin{aligned}
		\Delta U &= - \int (\vec{F}) \cdot d \vec{l} = U_2 - U_1 \\
		U_0 &= 0 \text{ when } y = 0 \\
		dU &= - \vec{F} \cdot d \vec{l}
	\end{aligned}\]
	\[\begin{aligned}
		&\Delta U = mgh \\
		&Wg = \int(\vec{F})d\vec{l} = -\Delta U
	\end{aligned}\]
\end{document}




























